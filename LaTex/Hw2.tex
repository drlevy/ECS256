%
\documentclass[10pt,a4paper]{article}
\usepackage[latin1]{inputenc}
\usepackage{amsmath}
\usepackage{amsfonts}
\usepackage{amssymb}
\usepackage{listings}
\usepackage{tabu}
\usepackage[width=14.00cm, height=25.00cm]{geometry}
\begin{document}

\section{Math Problem B}
Define $X_n$ as a bernoulli random variable such that,
\[X_n = \begin{cases} 
A & \text{bus arrival at time n} \\
NA & \text{no bus arrival at time n} \\
\end{cases}\]
We can then define a DTMC on \(X_n\) with transition probabilities,
\begin{gather*}
P(X_{t+1}=A|X_t=A)\\
P(X_{t+1}=NA|X_t=A)\\
P(X_{t+1}=A|X_t=NA)\\
P(X_{t+1}=A|X_t=NA)
\end{gather*}
We are interested in finding the pmf,
\begin{align*}
P(D_t=w)
\end{align*}
Representing the amount of time, $w$, a traveler has to wait for an arriving
bus given the traveler starts waiting at time $t$.\\
\\
For the case $P(D_t=0)$, a bus arriving when the traveler starts waiting,
either a bus appeared at time $t$ after there had been no bus at $t-1$ or
the buses arrived consecutively; i.e.,
\begin{align*}
P(D_t=0)=P(X_t=A|X_{t-1}=NA)+P(X_t=A|X_{t-1}=A)
\end{align*}
Using the independence of bus arrivals, we can then generalize the waiting
time from the initial time the traveler starts waiting to the time when a
bus finally arrives (for \(w\geq1\)),
\begin{gather*}
P(D_t=w)=P(A|NA)P(NA|NA)^{w-1}P(NA|NA)+P(A|NA)P(NA|NA)^{w-1}P(NA|A)\\
=P(A|NA)P(NA|NA)^{w-1}[P(NA|NA)+P(NA|A)]
\end{gather*}
\section{Math Problem C}
\subsection{Continuous Markov Chain}
The continuous Markov chain is defined for (i,j,k) where i represents the number of customers being served, j represents the queue for the manager, and k represents the number of people in the general queue. This model follows the model described on the blog. If the queue is empty and i is not equal to the number of servers available then any new customer goes immediately to a server with rate alpha*(1-p) and to the manager with alpha*p rate. The values of i,j,k are bounded by i + j + k $<=$ queue buffer. When the queue is full, no new customers can show up.

\subsection{Proportion of calls denied}
The proportion of calls denied is equal to the number of calls denied divided by the total number of calls. This is equal to the P(i + j + k = b AND we get a call) = P(i+j+k=b)*P(we get a call). The probability we are in a full state is equal to:\\
$\sum\limits_{i+j+k = b} \pi_{ijk}*P(call) = \sum\limits_{i+j+k = b} \pi_{ijk}*\frac{\alpha}{\alpha + \sigma + \omega + \mu}$
\subsection{Proportion of customers leaving due to impatience}
The proportion of customers leaving due to impatience is equal to the number of people would leave divided by the total number of people. The argument flows similar to above.\\
$\sum\limits_{k > 0 or j > 0} \pi_{ijk}*\frac{\omega}{\alpha + \sigma + \omega + \mu}*(k+j)$
\section{Simulation}


\end{document}


